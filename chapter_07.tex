\paragraph{Def 7.1. Sum of Subspaces}$\sum_{i=1}^nS_i:=\{\sum_{i=1}^n\mathbf{x}_i\mid \mathbf{x_i}\in S_i\}$

\paragraph{Def 7.3. Coset} $\overline{\mathbf{v}}:=\mathbf{v}+W$, or $[\mathbf{v}]:=\mathbf{v}+W$

\paragraph{Ex 7.4. Coset Relation} The following 4 are equivalent:
\begin{multicols}{2}
\begin{itemize}[noitemsep,nolistsep]
    \item $\mathbf{v} + W = W$, i.e. $\overline{\mathbf{v}}=\overline{\mathbf{0}}$;
    \item $\mathbf{v}\in W$;
    \item $\mathbf{v} + W \subseteq W$;
    \item $W \subseteq \mathbf{v} + W$.
\end{itemize}
\end{multicols}

\paragraph{Thm 7.5.} $\mathbf{v_2}-\mathbf{v_1}\in W \Rightarrow \overline{\mathbf{v}_1}=\overline{\mathbf{v}_2}$ 

\paragraph{Thm 7.8. Quotient Space} Quotient space of $V$ modulo $W$:
$V/W := \{\overline{\mathbf{v}}=\mathbf{v}+W\mid \mathbf{v}\in V\}$.
$$\begin{aligned}
+:V/W\times V/W &\rightarrow V/W \\
(\overline{\mathbf{v}_1}, \overline{\mathbf{v}_2}) & \mapsto \overline{\mathbf{v}_1} + \overline{\mathbf{v}_2} :=\overline{\mathbf{v_\text{1}+v_\text{2}}} \\
\times:F\times V/W &\rightarrow V/W \\
(a, \overline{\mathbf{v}_2}) & \mapsto a\overline{\mathbf{v}_1} := \overline{a\mathbf{v}_1}
\end{aligned}$$

\paragraph{Thm 7.9. QS Well-def} 1) Binary addition; 2) scalar mult are well defined; $V/W$ is a vector space over $F$ with $\mathbf{0}_{V/W}=\overline{\mathbf{0}_V}=\overline{\mathbf{w}}$.

\paragraph{Thm 7.10. L.T.} $\varphi: V_1\rightarrow V_2$ is L.T. iff
$$
\begin{aligned}
\varphi(\mathbf{v}_1 + \mathbf{v}_2) &= \varphi(\mathbf{v}_1) + \varphi(\mathbf{v}_2) & (\forall \mathbf{v}_i \in V)\\
\varphi(a\mathbf{v}) &= a\varphi (\mathbf{v}) & (\forall a \in F, \forall \mathbf{v}\in V)
\end{aligned}
$$
or $\varphi(a_1\mathbf{v}_1 + a_2 \mathbf{v}_2) = a_1 \varphi (\mathbf{v}_1) + a_2 \varphi (\mathbf{v}_2)$. In this case, $\varphi$ is a \textbf{linear operator}. 

\paragraph{Def. 7.10. Isomorphism} $\varphi: V_1\rightarrow V_2$ is isomorphism if $\varphi$ is a \textbf{bijective} L.T. 

\textbf{Alt Def.} $\operatorname{Ker}(\varphi) = \{\mathbf{0}_V\} \land \varphi(V)=W \Leftrightarrow \varphi \text{ is isomorphism}$. 

\paragraph{Def.} Let $\varphi: V\rightarrow W$:
\begin{itemize}[noitemsep,nolistsep]
    \item \textbf{Domain}: $\operatorname{dom}(\varphi) := V$;
    \item \textbf{Codomain}: $\operatorname{codomain}(\varphi):=W$;
    \item \textbf{Kernel}: $\operatorname{Ker}(\varphi):=\{\mathbf{v}\mid \varphi(\mathbf{v}) = \mathbf{0}_W\}\subseteq V$;
    \item \textbf{Range}: $R(\varphi):=\{\varphi(\mathbf{v})\mid \mathbf{v}\in V\}\subseteq W$.
\end{itemize}

\paragraph{Relation with Matrices} Let $T_A$ be L.T. assoc. with $A=(a_{ij})\in M_{m\times n}(F)$, then:
\begin{itemize}[noitemsep,nolistsep]
    \item $\operatorname{Ker}(T_A) =\operatorname{Null}(A) =\{X\in F_c^n\mid AX=\mathbf{0}\}\subseteq F_c^n$;
    \item $\operatorname{Null}(A)=\mathbf{0} \Leftrightarrow T_A\text{ injective}$;
    \item $(\dim R(A):=\operatorname{rank}(A)=m)\Leftrightarrow T_A \text{ surjective}$;
    \item $T_A \text{ isomorphism} \Leftrightarrow A \text{ invertible square}$.
\end{itemize}

\paragraph{Rmk 7.13. $\bigoplus$ v.s. QS} Let $V := U \oplus W$. $\varphi: U\rightarrow U/W$ where $\mathbf{u}\rightarrow\bar{\mathbf{u}}=\mathbf{u}+ W$ is an \textbf{isomorphism}.

\paragraph{Maps} Some Common Maps:
\begin{itemize}[noitemsep,nolistsep]
    \item \textbf{Zero Map}: $\mathbf{0}:V\rightarrow W$ with $\mathbf{v} \mapsto \mathbf{0}_W$;
    \item \textbf{Scalar Map}: $\alpha I_V:V\rightarrow V$ with $\mathbf{v}\mapsto \alpha I_V(\mathbf{v})= \alpha\mathbf{v}$
    \item \textbf{Inclusion Map}: $\iota : W\rightarrow V$ with $\mathbf{w}\mapsto \mathbf{w}$. This is \textbf{injective}.
    \item \textbf{Restriction Map}: $T:V\rightarrow W$, let $V_1\subseteq V$ and $\iota: V_1\rightarrow V$ be the inclusion map. Then: $T_{V_1}:=T\circ \iota: V_1 \rightarrow W$ with $\mathbf{v}\mapsto (T\circ \iota)(\mathbf{v}) = T(\iota(\mathbf{v})) = T(\mathbf{v})$;
    \item \textbf{Quotient Map}: $\gamma : V\rightarrow V/W$ with $\mathbf{v}\mapsto \overline{\mathbf{v}}$. This map is \textbf{surjective} with $\operatorname{Ker}(\gamma) = W$.
\end{itemize}

\paragraph{Mult.} Let $\varphi_1:V_1\rightarrow V_2$ and $\varphi_2:V_2\rightarrow V_3$, then $\varphi_2\varphi_1:=\varphi_2\circ\varphi_1$.

\paragraph{Thm 7.28. Img as Subspace} Let $\varphi: V\rightarrow W$ and $V_1$ a subspace of $V$. Then $\varphi(V_1)$ is a subspace of $W$.

\paragraph{Thm 7.29. Subspace V.S. Kernel}
\begin{itemize}[noitemsep,nolistsep]
    \item Let $\varphi: V\rightarrow U$, $\operatorname{Ker}(\varphi)$ is a subspace of $V$; 
    \item Let $W$ be a subspace of $V$, $\exists \varphi :V\rightarrow U$ s.t. $W = \operatorname{Ker}(\varphi)$. ($\varphi: V \rightarrow V/W$ with $\mathbf{v} \mapsto \overline{\mathbf{v}}$.)
\end{itemize}

\paragraph{Thm 7.39. $1^\text{st}$ Iso Thm} Let $\varphi: V \rightarrow U$ be LT. Then there is an isomorphism $\overline{\varphi}:V/\operatorname{Ker}(\varphi) \xrightarrow{\mathtt{\sim}} \varphi(V) \subseteq U$ where $\overline{\mathbf{v}}\mapsto \varphi(\mathbf{v})$ such that $\varphi=\overline{\varphi}\circ\gamma$ where $\gamma:V\rightarrow V/\operatorname{Ker}(\varphi)$ with $\mathbf{v}\rightarrow\overline{\mathbf{v}}$ is the quotient map, is a linear transformation. When $\varphi$ is surjective, we have the isomorphism: $\overline{\varphi}: V / \operatorname{Ker}(\varphi) \stackrel{\sim}{\rightarrow} U$.

\paragraph{Thm 7.40. Basis of QS} Let $W$ be a subspace of $V$ with basis $B_1$. Extend $B_1$ to a basis $B:=B_1\coprod \{\mathbf{w}_{r+1}, \cdots, \mathbf{w}_n\}$. Then, the cosets $\{\overline{\mathbf{w}_{r+1}}, \cdots, \overline{\mathbf{w}_n}\}$ is the basis of the QS $V/W$.

\paragraph{Thm 7.45. Dimension Thm} Let $\varphi: V\rightarrow W$ be LT between vector spaces over a field $F$. Then:
$$
\operatorname{dim}_F\operatorname{Ker}(\varphi) + \operatorname{dim}_F\varepsilon(V) = \operatorname{dim}_F V.
$$
If $A \in M_{m\times n}(F)$, then 
$$
\operatorname{nullity}(A) + \operatorname{rank}(A)=n
$$

\paragraph{Thm 7.46. $2^\text{nd}$ Iso Thm} Let $W_1$, $W_2$ be vector subspaces of $V$. The map 
$$\begin{gathered}\varphi:W_1 /\left(W_1 \cap W_2\right) \rightarrow\left(W_1+W_2\right) / W_2 \\\mathbf{w}+\left(W_1 \cap W_2\right)=\overline{\mathbf{w}} \mapsto \overline{\mathbf{w}}=\mathbf{w}+W_2\end{gathered}$$ is a well-defined isomorphism. By this, we have $\dim W_1 + \dim W_2 = \dim(W_1 + W_2) + \dim (W_1  \cap W_2)$.